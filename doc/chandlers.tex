\chapter{Writing your own handler class}

In previous chapters we have seen some of the handlers {\tt itools} provides
out of the box, and how they help us to work with resources (e.g. XML
documents).


This chapter will explain how to build your own handler classes, which will
let you to add support for file formats either standard or defined by yourself.
The explanation will be driven by an example.

Following up the previous chapter, we will extend our addressbook, which
will become a handler class. This means that the list of addressess (each
one with its last name, first name and telephone number) will be stored in
a resource, hence giving persistence to our addressbook.

\section{Step by step}

The process to write a handler class can be splitted in several steps, these
are:

\begin{enumerate}
  \item First the file format must be choosen or designed. If there is an
    standard file format that matches our requirements it could be wise to
    use it. Other option is to design your own.

    For example, if, in the context of a miltilingual application, we want
    to have a catalog that keeps sentences in one language and its
    translations to another language, then there are two obvious options,
    the PO file format from GNU gettext and the TMX standard ({\bf T}ranslation
    {\bf M}emory e{\bf X}change\footnote{http://www.lisa.org/tmx}).

    In other situations we may want to design our own format, this is the
    case we will illustrate in this chapter with the new version of
    the addressbook.

  \item The second step is to choose the handler class to use as base class.

    For instance, if we're going to write a handler for TMX the base handler
    class should be {\tt XML.Document}. If we want to write a handler for
    {\bf C}omma {\bf S}eparated {\bf V}alue files, then our class should
    inherit from {\tt Text}.

  \item Usually we will implement the method {\tt \_load()}, which is the
    responsible to parse the resource data and to build the data structure
    we want in memory.

  \item It is likely we will also write the method {\tt get\_skeleton()}
    which must return the default content of a resource.

  \item Finally we will develop the specific API that will enable us to
    work with our handler.
\end{enumerate}

\section{Example: Addressbook}

\subsubsection{}