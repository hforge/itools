\chapter{Writing file handler classes}

The chapter before we have learn about file handlers and how to use them,
now we are going to learn to write our own handler classes, what by the
way will help to solodify the concepts seen before.

The explanation will be driven by an example, we are going to write a task
tracker (what would be very handy to keep accounting of the {\tt itools}
bugs, if {\tt itools} had bugs of course). The code for the example can
be found in the directory {\tt examples/chapter5}.

\section{Functional scope}

Lets start by defining the functional scope of our task tracker. It is
going to be very simple, it will be a collection of tasks where every
task will have four fields:

\begin{itemize}
  \item {\em id}, it will be a natural number, starting from 0, which
    will uniquely identify each task.

  \item {\em title}, a short sentence describing the task.

  \item {\em description}, a longer description detailing the task.

  \item {\em state}, it maybe {\em open} (if the task has not been
    finished yet), or {\em closed} (if the task has been finished).
\end{itemize}

The task tracker will provide an API to manipulate the collection of
tasks: create a new task, see either the open or the closed tasks,
and close a task.


\section{The file format}

Now that we know what we want to do, we have to decide where and how
the information will be stored.

We will keep the tasks in a single text file, with a format somewhat
similar to the one used by the standards {\em vCard} and {\em iCal},
for example:

\begin{code}
    id:0
    title:Re-write the chapter about writing handler classes.
    description:A new chapter that explains how to write file handler
     classes must be written, it should go inmediately after the chapter
     that introduces file handlers.
    state:closed

    id:1
    title:Finish the chapter about folder handlers.
    description:The chapter about folder handlers needs much more work.
     For example the skeleton of folder handlers must be explained.
    state:open
\end{code}

Each task is separated from the next one by a blank line. Every field
starts by the field name follewed by the field value, both are separated
by colon. If a field value is very long it can be written in multiple lines,
where the second and next lines start by an space.

This very same file can be found in the examples directory with the name
{\tt itools.tt} name. Using our own filename extansion ({\tt tt}) will
prove useful as we will see later.


\section{Loading\dots}

The first draft of our handler class will be able to load (de-serialize)
the resource into a data structure on memory.


