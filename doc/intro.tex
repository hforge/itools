\chapter{Introduction}

\section{What is {\tt itools}?}

{\tt itools} is a Python\footnote{\tt http://www.python.org} library, it
groups a number of packages into a single meta-package for easier
development and deployment. The packages included are:

\begin{quote}
\begin{tabular}{lll}
  {\tt itools.catalog} & {\tt itools.ical} & {\tt itools.web} \\
  {\tt itools.datatypes} & {\tt itools.resources} & {\tt itools.workflow} \\
  {\tt itools.gettext} & {\tt itools.rss} & {\tt itools.xhtml} \\
  {\tt itools.handlers} & {\tt itools.schemas} & {\tt itools.xliff} \\
  {\tt itools.html} & {\tt itools.tmx} & {\tt itools.xml} \\
  {\tt itools.i18n} & {\tt itools.uri} & \\
\end{tabular}
\end{quote}

The Figure~\ref{Figure: packages} shows the dependencies between the
different packages. Below a brief description of each package (sorted
by alphabetical order):

\begin{figure}
  \center
  \includegraphics[width=\textwidth]{packages.eps}
  \caption{Dependency diagram}
  \label{Figure: packages}
\end{figure}

\begin{itemize}
  \item {\tt itools.catalog} -- An index and search engine. It provides
    full text indexing as well as keyword indexing; and boolean, phrase
    and range search.

  \item {\tt itools.datatypes} -- Type marshalers for basic types (integer,
    date, etc.) and not so basic types (filenames, XML qualified names, etc.).

  \item {\tt itools.gettext} -- Support for the PO and MO file formats of
    the GNU gettext tools.

  \item {\tt itools.handlers} -- Resource handlers are non persistent classes
    responsible to manage a resource (a file or a folder), they represent a
    resource as a data structure on memory and provide an API to work with
    the resource.

    Included are the resource handlers for folders, binary and text files,
    images, CSV, etc.

  \item {\tt itools.html} -- Support for HTML documents.

  \item {\tt itools.i18n} -- Miscellaneous tools for internationalization
    and localization (language negotiation, text segmentation, fuzzy
    matching, etc.).

  \item {\tt itools.ical} -- Support for the iCalendar standard (RFC 2445).

  \item {\tt itools.resources} -- An abstraction layer over resources (files
    and folders), provides a the same programming interdace accross different
    storages.

  \item {\tt itools.schemas} -- Infrastructure for datatypes schemas. Support
    for Dublin Core included.

  \item {\tt itools.tmx} -- Support for the industry standard
    {\em Translation Memory eXchange}.

  \item {\tt itools.uri} -- Support for {\em Uniform Resource Indentifiers}
    (RFC 2396).

  \item {\tt itools.web} -- Abstract programming interface for web
    applications. Proof-of-concept implementations for the CGI protocol
    and an standalone server.

  \item {\tt itools.workflow} -- Programming interface to implement workflow
    in applications.

    Workflows are represented as automatons, objects can move from one state
    to another through transitions, classes can add specific semantics to
    states and transitions.

  \item {\tt itools.xhtml} -- Support for XHTML documents.

  \item {\tt itools.xliff} -- Support for the industry standard {\em XML
    Localisation Interchange File Format}.

  \item {\tt itools.xml} -- Support for XML ({\em eXtensible Markup Language}).

    Includes an intuitive event driven parser, a DOM-like representation of
    XML documents, and the {\bf S}imple {\bf T}emplate {\bf L}anguage.

\end{itemize}


\section{Highlights}

\subsection{The resource-handler model}

Of everything in {\tt itools} I am probably most proud of the the three
sub-packages {\tt itools.uri}, {\tt itools.resources} and
{\tt itools.handlers}, which make up what I call the {\em resource-handler}
model.

There is a linear relationship of dependency between these modules, {\tt uri}
depends on nothing but Python, {\tt resources} depends on {\tt uri}, and
{\tt handlers} depends on {\tt resources}.

This allows to use these modules on a flexible way. For example, you can
just use {\tt itools.uri} (ignoring the others) to benefit from a higher
level API to work with URIs than that provided by the Standard Library
({\tt urlparse}).

Or you may use {\tt itools.resources} if you want to benefit from an
abstraction layer over the storage. This is to say, if you want to be able
to manipulate many resources stored in different systems and accessed
through different protocols with the same, consistent, rich API.

You may even take advantage of {\tt itools.handlers} without fully
understanding the model behind. For example, you could use some of the
handlers {\tt itools} offers out-of-the-box for standard file formats
like CSV, PO or XHTML, to simplify your live if you ever need to work
with one of these formats.

But, in order to exploit everything {\tt itools} has to offer to its limit,
you may choose to base part or all of your application architecture on the
{\em resource-handler} model, a model heavily influenced by the filesystem
and the Web. In this situation the three packages ({\tt uri}, {\tt resources}
and {\tt handlers}) show themselves as three different components with
a distinct role in the architecture:

\begin{itemize}
  \item {\tt itools.uri} identifies and locates resources, wherever they
    are. Altogether with {\tt itools.resources} it effectively enables your
    information system to be distributed through an heterogeneous base.

  \item {\tt itools.resources} provides persistency to your data, this is
    to say, resources are where the data is stored.

  \item and {\tt itools.handlers} is where the logic lives. The essential
    characteristic of a resource handler (or handler for short) is that it is
    non-persistent, instead it is associated with a resource, whose content
    it is responsible to manage.
\end{itemize}

The first chapters of this document will cover these sub-packages,
{\tt itools.uri}, {\tt itools.resources} and {\tt itools.handlers} with
detail.

\subsection{eXtensible Markup Language}

The {\tt itools.xml} package depends on {\tt itools.handlers}, this is its
first advantage. It represents an XML document as an DOM like tree. But it
also provides support for schemas, what allows to easily build higher
level data structures.

The package also provides handlers for specific document types out-of-the-box,
most notably XHTML and HTML (even if HTML is not XML, it shares a lot with
XHTML).


\subsection{The Simple Template Language}

The Simple Template Language is included in the {\tt itools.xml} package,
but it is important enough to deserve its own chapter in the documentation.

Its design goals are:

\begin{itemize}
  \item Truly separate logic and presentation. Even Python expressions are
    not allowed within the template.

  \item Really simple. It can be mastered in half a day.

  \item Damn fast (easy to achieve through simplicity).

  \item Secure. Even non-trusted users could write templates without risk,
    because code is not allowed within the template.
\end{itemize}

And the key idea behind, is to make it a {\em descriptive} language.


\subsection{Workflow}

The sub-package {\tt itools.workflow} (once known as {\em flux}) is the
oldest code in {\tt itools}. It does not depend on anything but the
Standard Library, and don't puts any restriction on the storage. This
makes it very easy to combine with other frameworks.

There is a chapter exclusively dedicated to it.

\subsection{Internationalization and Localization}

The {\tt itools.i18n} package provides a wide range of tools for
internationalization and localization of both software and data. From
message extraction to language negotiation, through text segmentation,
fuzzy matching or algorithms to guess the language a text is written in.

A chapter is devoted to this topic.


\subsection{Index and Search}

The {\tt itools.catalog} sub-package provides an index and search engine.
Though still young, it already provides full text indexing, boolean queries
and results sorted by weight.


\section{Project status}

Currently {\tt itools} is under active development. At the time of this
writing the last version available is {\bf 0.10}, what it provides is
the scope of this document. Here I want to offer a glance about what is
comming next:

\begin{itemize}
  \item SQL. Relying on third party products, we will explore relational
    databases and see what we can do to simplify its use. Specifically
    the goal is to provide a programming interface that seemlessly integrates
    with the rest of {\tt itools}.

  \item Standards support. {\tt itools} is strongly focused on the
    implemetation of standards. While we will add new standards to the
    collection, in the short-term the focus will be on improving the
    support we already provide for some of them, just two spotlights:

     \begin{itemize}
       \item A native parser for XML would increase the control we have of
         the parsing process, to allow things like parsing fragments. Also,
         some rough corners of the XML API must be polished.

       \item CSV is a very common format, hence to have a very good
         programming interface is more of a priority. Probably a new
         package, {\tt itools.csv}, will be implemented.
     \end{itemize}

  \item Performance. While performance is already one of the strong points
    of {\tt itools}, as the {\em Simple Template Language} illustrates, it
    is not enough:

    \begin{itemize}
      \item One thing we need is better benchmarks, so we can better
        measure how {\tt itools} compares to other alternatives, specially
        for the catalog.

      \item Support for lazy load will be added to the most important handler
        classes.

      \item While profiling the CPU usage is easy with the Python profiler,
        the lack of similar tools to measure the memory consumption makes
        it hard to optimize this aspect. Anyway some infrastructure must be
        put in place.
    \end{itemize}

  \item Documentation, is critical. The target for this document is to cover
    all of {\tt itools}. Also some changes in the structure may arise, one
    idea is have one complete reference guide, plus a collection of
    tutorials.

  \item Lower the entry point, improve the development process. Today we
    use {\em tla 1} to manage the sources, it is great when you master it,
    but is hard to get there. It looks like the new version, {\tt tla 2},
    is addressing this problem; other alternatives like bazaar are worth
    considering.

    To keep improving the development process of {\tt itools} is a constant
    we don't stop working on.

\end{itemize}

For further information you are welcome to the {\tt itools} mailing list;
the bugtracker is available to report bugs and suggestions; the sources
can be browsed through the web; and don't forget to check the web site
for last downloads and news:

\begin{itemize}
  \item Mailing list, {\tt http://in-girum.net/mailman/listinfo/ikaaro}
  \item Bug Tracker, {\tt http://in-girum.net/cg-bin/bugzilla/index.cgi}
  \item Brows the sources, {\tt http://in-girum.net/cgi-bin/archzoom}
  \item The web site, {\tt http://www.ikaaro.org/itools}
\end{itemize}



\section{Installation}

{\tt itools} requires Python 2.4, earlier versions are not supported.

The last version of {\tt itools} can be downloaded from
{\tt http://www.ikaaro.org}. It is distributed as tarball, download and
unpack it somewhere; then install with {\em distutils}, type:

\begin{code}
    $ python setup.py install
\end{code}

Be sure to have the right permissions.


\section{About this document}

This paper is the official documentation of {\tt itools}. It is
addressed to Python developers. Though it may be useful to software
architects in general, as some ideas exposed here may be found
interesting\footnote{For example, {\tt itools.workflow} inspired XXX
to write a similar engine in Java, see {\tt http://XXX}}.

It is convenient to have some basic skills with the Python programming
language to fully understand this document. Probably the best introduction
to Python is the official tutorial:

\begin{quote}
  {\tt http://python.org/doc/2.4.1/tut/tut.html}.
\end{quote}

This document touches many different technologies and standards, such as XML.
References will be given in the relevant chapters.

Currently the document covers around 60\% of the {\tt itools} package.
There are also two appendixes, one explaining the coding style {\tt itools}
is written in, another one introducing the use of {\em GNU arch}. Both are
specially addressed to those that want to contribute back to the main tree.
